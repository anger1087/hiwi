
\chapter{Waves in a Discrete World}

{\bf Comprehension questions}
\begin{enumerate}
\setcounter{enumi}{0}
\item
Describe concepts how important for representing space-dependent functions in a discrete way. 
\begin{enumerate}
\item[]
\item[]
\item[] 
\item[]
\item[] 
\end{enumerate}
\item
Explain the concept of 2D, 2.5D, and 3D simulations. What problems can arise for $<$3D simulations when comparing with observations?
\begin{enumerate}
\item[]
\item[]
\item[] 
\item[]
\item[] 
\end{enumerate}
\item
Explain the concepts of structured and unstructured meshes\ix{mesh}.  Give examples. 
\begin{enumerate}
\item[]
\item[]
\item[] 
\item[]
\item[] 
\end{enumerate}
\item
Illustrate the differences between regular meshes\ix{mesh!regular} in various coordinate systems: cartesian, cylindrical, spherical. What are the consequences for simulation problems?
\begin{enumerate}
\item[]
\item[]
\item[] 
\item[]
\item[] 
\end{enumerate}
\item
What is a {\it cubed sphere}\ix{cubed sphere} ?
\begin{enumerate}
\item[]
\item[]
\item[] 
\item[]
\item[] 
\end{enumerate}
\item
Explain the concept of Delauney\ix{Delauney triangulation} triangulation and Voronoi cells\ix{Voronoi cells}.
\begin{enumerate}
\item[]
\item[]
\item[] 
\item[]
\item[] 
\end{enumerate}
\item
Discuss pros and cons of structured vs. unstructured grids\ix{mesh!structured}. 
\begin{enumerate}
\item[]
\item[]
\item[] 
\item[]
\item[] 
\end{enumerate}
\item
What are adaptive meshes\ix{adaptive mesh refinement}? Can they be used in seismology?
\begin{enumerate}
\item[]
\item[]
\item[] 
\item[]
\item[] 
\end{enumerate}
\item
Give reasons why the generation of meshes\ix{mesh generation} is relevant for seismological problems. Give examples.
\begin{enumerate}
\item[]
\item[]
\item[] 
\item[]
\item[] 
\end{enumerate}
\item
What are the basic models for parallel computers\ix{parallel computing!hardware}?
\begin{enumerate}
\item[]
\item[]
\item[] 
\item[]
\item[] 
\end{enumerate}
\item
What is the most common model of parallelisation for seismic wave propagation and why? 
\begin{enumerate}
\item[]
\item[]
\item[] 
\item[]
\item[] 
\end{enumerate}
\item
Explain the concepts of strong and weak scaling. \ix{parallel computing!scaling}
\begin{enumerate}
\item[]
\item[]
\item[] 
\item[]
\item[] 
\end{enumerate}
\item
Find some current supercomputers on the internet and extract the main specifications (e.g., number of processors, memory, peak performance, etc.). 
\begin{enumerate}
\item[]
\item[]
\item[] 
\item[]
\item[] 
\end{enumerate}
\item Class exercise: Every student gets an integer number starting with 0. This number denotes the processor. Each students writes four numbers on a page. Perform the following tasks:
\begin{itemize}
	\item Single-Instruction-Multiple-Data: The tutor tells all students to multiply each number by 5.
	\item Embarrassingly parallel problem: Add the first three numbers and subtract the fourth. 
	\item Circular shift operation: Pass the first number to your right neighbour. If there is none, pass it to the far left neighbour.  
	\item Global reduce: Find the maximum value of all initial 4 numbers of all processors. Processor 0 speaks out the maximum value loudly. 
	\item Global distribute: The tutor gives 2 numbers to processor 0. Processor 0 distributes these 2 numbers to all processors.    
	\item Extend these exercises to your liking. 
\end{itemize}
\begin{enumerate}
\item[]
\item[]
\item[] 
\item[]
\item[] 
\end{enumerate}
\end{enumerate}

\noindent {\bf Theoretical problems}
\begin{enumerate}
\setcounter{enumi}{14}
\item
Classify the following partial differential equations in terms of elliptical, hyperbolic, or parabolic problems:\ix{partial differential equations!classification}
\be
\begin{split}
u_{xx} + 2 c u_{xt} + c^2 u_{tt} &= \ 0 \\
x u_{xx} - 4 u_{xt} &= \ 0 \\
u_{xx}  - 6 u_{xt} + 12 u_{tt} &= \ 0 
\end{split}
\ee 
\begin{enumerate}
\item[]
\item[]
\item[] 
\item[]
\item[] 
\end{enumerate}
\item
Use Eq.~\ref{m_scaling} to find out what fraction of the code needs to be parallel to achieve a speed up of 10000 for 20000 processors. 
\begin{enumerate}
\item[]
\item[]
\item[] 
\item[]
\item[] 
\end{enumerate}
\item
You want to simulate a physical domain of size (or side length) 1000km with a grid distance of 1km. Estimate the required memory of one space-dependent field (double precision) in 1D, 2D, and 3D. 
Compare with the RAM of your smart phone, laptop, and with the specifications of current supercomputers. Discuss the results.\ix{parallel computing!memory}
\begin{enumerate}
\item[]
\item[]
\item[] 
\item[]
\item[] 
\end{enumerate}
\end{enumerate}

\noindent {\bf Programming exercises}
\begin{enumerate}
\setcounter{enumi}{17}
\item
Write a program (e.g., Matlab, Python) that generates arbitrary point clouds. Triangulate them using the Delauney method. Calculate and visualise the corresponding Voronoi cells. Find appropriate libraries to carry out the tasks. 
\begin{enumerate}
\item[]
\item[]
\item[] 
\item[]
\item[] 
\end{enumerate}
\item
Distributed data: Write a small parallel program using (e.g., Fortran/MPI or pyMPI). Define a matrix A(2000,2000) and distribute it on $n$ processors. Initialise it with random numbers and extract minimum and maximum values. Perform operations on the matrix in a loop and time the operations. Compare to the serial case $n=1$. 
 and distributed task parallelism. \ix{parallel computing!MPI}
 \begin{enumerate}
\item[]
\item[]
\item[] 
\item[]
\item[] 
\end{enumerate}
\item
Task parallelism: Write a small parallel program using (e.g., Fortran/MPI or pyMPI). Load a seismogram trace. In one processor calculate filtered seismograms (e.g., looping through lowpass filters with various corner frequencies). In a second processor perform an equal number of subsequent auto-correlations. Time the parallel and serial codes and compare the results. Note: If you use Python the ObsPy library offers tools for seismic data processing.   
\begin{enumerate}
\item[]
\item[]
\item[] 
\item[]
\item[] 
\end{enumerate}
\item
Search for open-source mesh generators (e.g., $MeshPy$), invent some simple geometries, generate $vtk$ (visualisation toolkit) files and visualise them (e.g. with $paraview$).
\begin{enumerate}
\item[]
\item[]
\item[] 
\item[]
\item[] 
\end{enumerate}
\item
Install the $ObsPy$ Python library (www.obspy.org). Follow the tutorials and investigate the potential to access and process observed and/or simulated seismic data. 
\begin{enumerate}
\item[]
\item[]
\item[] 
\item[]
\item[] 
\end{enumerate}
\item
Use $ObsPy$ to download data from any seismic station you are interested in for the M9.1 Tohoku-Oki earthquake March 11,2011. Save the data using the $ASDF$ format (seismic-data.org). Explore the provenance options.   \ix{parallel computing!provenance}
\begin{enumerate}
\item[]
\item[]
\item[] 
\item[]
\item[] 
\end{enumerate}
\end{enumerate}