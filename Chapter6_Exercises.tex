
\chapter{The Finite-Element Method}

{\bf Comprehension questions}
\begin{enumerate}
\item In which community was the finite-element method primarily developed? Give some typical problems. 
\begin{enumerate}
\item[]
\item[]
\item[] 
\item[]
\item[] 
\end{enumerate}
\item What are weak\ix{weak form} and strong\ix{strong form} forms of partial differential equations? Give examples. 
\begin{enumerate}
\item[]
\item[]
\item[] 
\item[]
\item[] 
\end{enumerate}
\item Discuss the pros and cons of the finite-element method vs. low-order finite-difference methods. 
\begin{enumerate}
\item[]
\item[]
\item[] 
\item[]
\item[] 
\end{enumerate}
\item Present and discuss problem classes that can be handled well with the finite-element method. Compare with problems better handled with other methods. 
\begin{enumerate}
\item[]
\item[]
\item[] 
\item[]
\item[] 
\end{enumerate}
\item Compare the spatial discretisation strategies of finite-element and finite-difference methods. 
\begin{enumerate}
\item[]
\item[]
\item[] 
\item[]
\item[] 
\end{enumerate}
\item Describe the derivation strategy  of finite-element shape functions\ix{shape functions}.
\begin{enumerate}
\item[]
\item[]
\item[] 
\item[]
\item[] 
\end{enumerate}
\item Discuss qualitatively (use sketches) the use of basis functions. Compare with the interpolation properties of the pseudospectral method. 
\begin{enumerate}
\item[]
\item[]
\item[] 
\item[]
\item[] 
\end{enumerate}
\item Is the finite-element method a global or a local scheme? Explain.
\begin{enumerate}
\item[]
\item[]
\item[] 
\item[]
\item[] 
\end{enumerate}
\item Why does the finite-element method require the solution of a (possible huge) system of linear equations? What is the consequence for parallel computing?
\begin{enumerate}
\item[]
\item[]
\item[] 
\item[]
\item[] 
\end{enumerate}
\item Why is the classic linear finite-element method not so much used for seismological research today?
\begin{enumerate}
\item[]
\item[]
\item[] 
\item[]
\item[] 
\end{enumerate}
\item
Explain the benefits of the finite-element method with respect to Earth models with complex geometries. 
\begin{enumerate}
\item[]
\item[]
\item[] 
\item[]
\item[] 
\end{enumerate}
\end{enumerate}

\noindent {\bf Theoretical problems}
\begin{enumerate}
\setcounter{enumi}{10}
\item
The  advection equation\ix{advection equation} is  
\be
 \partial_t q(x,t) + c(x) \partial_x q(x,t) = 0
  \nonumber
\ee
where $q(x,t)$ is the scalar quantity to be advected and $c(x)$ is the advection velocity. Write down the weak form of this equation and perform integration by parts. 
What happens to the anti-derivative? Does it cancel out at the boundaries like in the 1D elastic wave equation? Note: This is the point of departure for the discontinuous Galerkin method). 
\begin{enumerate}
\item[]
\item[]
\item[] 
\item[]
\item[] 
\end{enumerate} 
\item  
Are the linear basis functions 
\begin{equation}
	\varphi_i (x) = 
	\begin{cases} \frac{x - x_{i-1}}{x_i - x_{i-1}} \;  \text{for} \;  x_{i-1} < x \leq x_i \\ 
	\frac{x_{i+1} - x}{x_{i+1} - x_i} \; \text{for} \; x_i < x < x_{i+1} \\
	\phantom{-} 0 \; \; \; \; \; \; \; \; \; \; \text{elsewhere}
	\end{cases}
	\nonumber
\end{equation}
orthogonal?
\begin{enumerate}
\item[]
\item[]
\item[] 
\item[]
\item[] 
\end{enumerate}
\item
Derive the formulae for the calculation of the mass matrix\ix{finite-element method!mass matrix} elements (Eq.~\ref{eq_M1} and Eq.~\ref{eq_M2}) by sketching the integration interval, and the corresponding basis functions. 
\begin{enumerate}
\item[]
\item[]
\item[] 
\item[]
\item[] 
\end{enumerate}
\item
Calculate all entries of the stiffness matrix\ix{finite-element method!stiffness matrix} $ K_{i j} = \int_D \mu \partial_x \; \varphi_i \;  \partial_x \; \varphi_j$ for a static elastic problem with $\mu =70$GPa and $h = 1$m for a problem with $n=5$ degrees of freedom. 
\begin{enumerate}
\item[]
\item[]
\item[] 
\item[]
\item[] 
\end{enumerate}
\item
A finite-element system has the following parameters: Element sizes $h=[1, 3, 0.5, 2, 4]$, density $\rho = [2, 3, 2, 3, 2]$kg/m$^3$. Calculate the entries of the mass matrix given by   $M_{ij} = \int_D \rho \; \varphi_i \; \varphi_j \; dx$ using linear basis functions.
\begin{enumerate}
\item[]
\item[]
\item[] 
\item[]
\item[] 
\end{enumerate}
\item
H-adaptivity\ix{h-adaptivity}. For the  simulation with varying velocities and element size with parameters given in Table~\ref{tab_fe_simulation_h}, calculate the time step required for $\epsilon=0.5$ in each of the subdomains. Remember that the stability criterion is $\epsilon=c_{max} dt/dx$ where $c_{max}$ is the maximum velocity in the entire physical domain. Discuss the result. 
\begin{enumerate}
\item[]
\item[]
\item[] 
\item[]
\item[] 
\end{enumerate} 
\item
Follow the approach of the derivation of shape functions and derive the cubic case\ix{basis function!cubic} in 1D: $u(x)=c_1 + c_2 \xi + c_3 \xi^2 + c_4 \xi^3$. What are key differences to quadratic and linear cases?   
\begin{enumerate}
\item[]
\item[]
\item[] 
\item[]
\item[] 
\end{enumerate}
\item
Derive the quadratic shape\ix{basis function!quadratic}  functions $N(\xi,\eta)$ for 2D triangles with the following node points: 
\be
\begin{split}
&P_1(0,0), P_2(1,0), P_3(0,1),\\
&P_4(1/2,0), P_5(1/2, 1/2), P_6(0, 1/2)
\end{split}
\nonumber
\ee
Note: Use Python (or other program) to solve the linear system of equations. 
\begin{enumerate}
\item[]
\item[]
\item[] 
\item[]
\item[] 
\end{enumerate}
\item
Derive the quadratic shape functions $N(\xi,\eta)$ for 2D rectangles with the following node points: 
\be
\begin{split}
&P_1(0,0), P_2(1/2,0), P_3(1,0), P_4(1,1/2), \\
&P_5(1, 1), P_6(1/2, 1), P_7(0,1), P_8(0,1/2)
\end{split}
\nonumber
\ee
\begin{enumerate}
\item[]
\item[]
\item[] 
\item[]
\item[] 
\end{enumerate}
\item
Derive the derivative matrix\ix{differentiation matrix}  $\textbf{D}$  for the finite-difference based second derivative (Eq.~\ref{eq_diffmat}). Show that when applied to a vector $\textbf{u}$ that contains an appropriate function (e.g., a Gaussian, $\sin$ function) you obtain an approximation of its derivative. 
\begin{enumerate}
\item[]
\item[]
\item[] 
\item[]
\item[] 
\end{enumerate}
\end{enumerate}

\noindent {\bf Programming Exercises}
\begin{enumerate}
\setcounter{enumi}{20}
\item
Write a computer program that solves the 1D static elasticity problem (Eq.~\ref{eq_static_solution}) using finite elements. Also code the finite-difference based relaxation problem\ix{relaxation method} (Eq.~\ref{eq_relax}) and compare the results. Reproduce Fig.~\ref{fig_fe1d_static}. Extend the formulation to arbitrary element sizes.
\begin{enumerate}
\item[]
\item[]
\item[] 
\item[]
\item[] 
\end{enumerate}
\item
Code the 1D elastic wave equation using finite elements (Eq.~\ref{eq_fe_solution}).  Determine numerically the stability limit and compare with the finite-difference solution. 
Implement the analytical solution for the homogeneous case (note: it is  the same as the 1D acoustic wave equation). Compare the numerical dispersion behaviour of the finite-element method with the corresponding low- (or high-) order finite difference method.
\begin{enumerate}
\item[]
\item[]
\item[] 
\item[]
\item[] 
\end{enumerate}
\item
Initialise a strongly heterogeneous velocity model with spatially varying element size. Try to match the results with a regular grid finite-difference implementation of the same model. Discuss the two approaches in terms of time step, run time, memory usage. 
\begin{enumerate}
\item[]
\item[]
\item[] 
\item[]
\item[] 
\end{enumerate}
\item
Plot the high-order 2D shape functions derived in the theoretical problems above.  
\begin{enumerate}
\item[]
\item[]
\item[] 
\item[]
\item[] 
\end{enumerate}
\item
Derive a finite-difference based centred differentiation matrix for the 1st derivative. Implement the 1D elastic wave equation in matrix form. Compare with the 1D finite-element implementation. 
\begin{enumerate}
\item[]
\item[]
\item[] 
\item[]
\item[] 
\end{enumerate}
\end{enumerate}